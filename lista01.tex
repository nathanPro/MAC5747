\documentclass[10pt,reqno,a4paper]{article}
\usepackage[utf8]{inputenc}
\usepackage[T1]{fontenc}

\usepackage{fixltx2e}
\usepackage{microtype}
\usepackage[letterpaper,hmargin=1in,vmargin=1in]{geometry}
\usepackage[brazil]{babel}
\usepackage{indentfirst}
\usepackage{amssymb}
\usepackage{mathtools}          % already includes amsmath
\usepackage{amsthm}
\usepackage{bbm}
\usepackage{graphicx}
\usepackage{enumerate}
\usepackage{ifthen}
\usepackage{color}
\usepackage[colorinlistoftodos]{todonotes}
\usepackage{algpseudocode}
\usepackage{verbatim}

\title{Geometria Computacional: Lista 1}
\author{Nathan Benedetto Proença}
\date{20 de agosto de 2018}

\newcommand{\tdots}{\,.\,.\,} % in place of \ldots
\DeclarePairedDelimiter{\abs}{\lvert}{\rvert}
\DeclarePairedDelimiter{\norm}{\lVert}{\rVert}
\DeclarePairedDelimiter\curly{\{}{\}}
\DeclarePairedDelimiter\paren{(}{)}
\DeclareMathOperator{\Lip}{Lip}
\newcommand{\eps}{\varepsilon}
\newcommand{\isdef}{\coloneqq}
\newcommand{\C}{\mathcal{C}}
\newcommand*{\Reals}{\mathbb{R}}
\newcommand*{\Naturals}{\mathbb{N}}
\begin{document}
\maketitle

\section*{Exercício 2}
Sendo sincero, eu não entendi exatamente o que o professor Sperto ganhou,
mesmo desconsiderando o erro que ele cometeu. De qualquer maneira, o maior
problema com esta abordagem é que ela abre mão das vantagens da técnica de
divisão e conquista. No melhor caso, torna o algoritmo $O(n^2)$, e no
pior pode fazer com que ele não termine.

Para ver que este é o caso, basta pensar no caso onde todos os pontos estão
na mesma reta. Se, como descrito no enunciado, todos forem para o conjunto E,
então teremos que $\abs{E} = n$. Assim, na próxima iteração teremos a mesma
entrada, o que irá causar um laço infinito. Se o algoritmo deixar um ponto no
conjunto $D$, seu comportamento será quadrático.

\section*{Exercício 4}

A ideia é relativamente simples. Suponho que os pontos $E$, da esquerda,
e os pontos $D$, da direita, estão ordenados por $Y$. Irei processar
os pontos de $E$ em ordem crescente de $Y$. Quando um ponto de $D$ tiver uma
coordenada $Y$ tão pequena ao ponto dele não ser comparado com o ponto de $E$
que estou processando, posso garantir que ele não precisará ser comparado com
nenhum outro ponto de $E$ ainda não processado --- pois todos estes tem coordenada
$Y$ maior. Assim, posso descartar ele. Desta maneira, consigo encontrar o começo
do intervalo dos pontos com os quais irei comparar meu ponto em tempo $O(1)$
amortizado, o que torna o algoritmo linear.

Escrevi o algoritmo em inglês por não conseguir traduzir a biblioteca do
\LaTeX \; para português.
\\

\begin{algorithmic}
\Require $D$ and $E$ are sorted in non-decreasing order by $Y$.
\Require $\eps$ is the minimum between the smallest distance found in $D$
and the smallest distance found in $E$.
\Require $n$ is the size of $D$, and $m$ is the size of $E$.
\Require $D$ and $E$ have already been filtered for points with $X$ coordinate
too far.
\Function{Combine}{$X, Y, E, n, D, m, \eps$}
    \State $i \gets 1$
    \For{from $k \gets 1$ to $n$}
        \While{$i \leq m$ and $\eps < Y[E[k]] - Y[D[i]]$}
            \State $i \gets i + 1$
        \EndWhile

        \State $j \gets i$
        \While{$j \leq m$ and $\abs{Y[D[j]] - Y[E[k]]} < \eps$}
                \State $\eps \gets \min (\eps, \texttt{Dist}(E[k], D[j]))$
        \EndWhile
    \EndFor\\
    \Return $\eps$
\EndFunction
\end{algorithmic}
\end{document}
