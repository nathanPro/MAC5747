\documentclass[10pt,reqno,a4paper]{article}

\usepackage[brazil]{babel}
\usepackage[utf8]{inputenc}
\usepackage[T1]{fontenc}

\usepackage{amssymb}
\usepackage{mathtools}
\usepackage{amsthm}
\usepackage{bbm}
\usepackage{enumerate}
\usepackage{microtype}
\usepackage{lmodern}
\usepackage{suffix}
\usepackage{notation}

\title{Triangulação de Delaunay}
\author{Nathan Benedetto Proença}
\date{7 de agosto de 2017}

\begin{document}
\maketitle
\section{O Algoritmo}

O algoritmo implementado é o descrito no capítulo 9 do livro dos holandeses,
e que foi discutido em aula. Isso facilitou muito a implementação, por me dar
uma descrição clara da visão geral do que é feito.

A principal ideia é fazer uma construção incremental, que mantém triangulações
parciais. Para cada novo ponto, é necessário encontrar qual triângulo o contém
e subdividí-lo. Além disso, é necessário ``flipar'' as arestas que deixam de
pertencer à triangulação. Uma análise probabilística permite concluir que
este algoritmo tem complexidade esperada $O(n\lg n)$, e que o consumo de tempo
é dominado pelo problema de localização na estrutura de dados.

A estrutura de dados que mantive é um DAG, onde cada nó é um triângulo de
alguma triangulação parcial que construi. Mantive dois invariantes:

\begin{enumerate}
    \item Um triângulo está na triangulação atual se e somente se ele
        não contém filhos no DAG.
    \item Para todo triângulo que não está na triangulação atual, a união da
        área dos filhos dele no DAG cobre a área do triângulo.
\end{enumerate}

Assim, a busca se torna extremamente simples. Começando da raíz do DAG,
enquanto eu estiver num nó que tem filhos, eu avanço para o primeiro nó
que contém o ponto que estou procurando. Note que a busca \emph{não trata
nenhum caso especial}. Ela retorna algum triângulo da triangulação atual
que contém o vértice atual. Decidi implementar desta maneira para diminuir
o trabalho na parte do algoritmo que eu sei que domina a complexidade.

Infelizmente isto não foi o bastante para tornar meu algoritmo eficiente,
pois acabei esbarrando em outro gargalo na implementação.

De qualquer maneira, com um triângulo que contém o ponto, posso checar
se ele já é um dos vértices, se ele está em alguma aresta, ou se ele
está no interior do triângulo. Tratei cada caso separado.

Tive duas dificuldades na implementação: uma teórica e uma prática. A
implementação descrita supõe que existem \emph{pontos no infinito}, sem
definir precisamente como manipulà-los. A outra dificuldade foi em manter
para todos os triângulos quais faces eram adjacentes à eles.

\section{Pontos no infinito}

Resolvi abordar a forma de representar pontos no infinito de uma maneira
mais teórica, começando com como definir o que eu queria dizer com
``pontos no infinito''.

Senti essa necessidade para evitar a abordagem ``chutar um valor grande o
suficiente''. Tenho dois problemas com esta forma de resolver a questão:
ela não é robusta, e piora minha situação numericamente.

A posição dos pontos é vital para as decisões tomadas pelo algoritmo,
principalmente na escolha de quais arestas devem ser invertidas. Não basta
ter 3 pontos fora do fecho convexo da entrada, e tampouco parece ser o
bastante ter um triângulo que contenha o fecho convexo da entrada.
Resolvi modificar a representação dos pontos, e não calcular o quão grande
deveriam ser as coordenadas para não interferir na resposta.

Além disso, escolher um valor muito grande \emph{me cria problemas numéricos}.
Encontrei na Wikipedia a referência para um paper do Guibas e do Stolfi que
mostrava como fazer o teste de pertencimento de um ponto ao círculo definido por
3 pontos através de um determinante. Com isso, consigo implementar meu projeto
sem aritmética de ponto flutuante. Por trabalhar em Python, estou já com
números inteiros de preciso arbitrária. Logo, dilatar os pontos por uma
constante desnecessariamente grande torna os números que estou trabalhando
maiores e prejudica minha eficiência de maneira desnecessária.

Logo, como representei os pontos? Seja $\alpha$ um número real positivo,
e seja $u$ um ponto no plano. O ponto $\alpha \cdot u$ é o mesmo ponto dilatado
a partir da origem. Se interpretar um ponto no infinito como \emph{comportamento
de $\alpha \cdot u$ para $\alpha$ suficientemente grande}, consigo implementar
minhas primitivas de maneira apropriada.

Por exemplo, seja $\alpha$ um real positivo, e $u, v$ e $w$ pontos no plano.
Irei primeiro deduzir o teste de esquerda. Note que, se denotarmos
$a \wedge b \coloneqq \texttt{area2}(a, b)$, temos que

\[\texttt{Esquerda}^+(a,b,c) \iff (b - a) \wedge (c - a) > 0.\]

Mais ainda, note que como $a \wedge a = 0$ para todo $a$, temos também
que $a \wedge b = - b \wedge a$. Assim, podemos concluir que

\[(b - a) \wedge (c - a) = a \wedge b + b \wedge c + c \wedge a.\]

Aqui o problema se resolve: Note que $a \wedge b$ é um determinante. Como
quero tratar os pontos de maneira uniforme, posso ter todos os 4 casos onde
$a$ ou $b$ são ou não pontos no infinito. Contudo, em todos os casos,
posso tratar \emph{a fórmula resultante como um polinômio em $\alpha$}.

Se, por exemplo, $a$ é um ponto no infinito, podemos escrever $a = \alpha \bar{a}$,
e afirmar que
\[a \wedge b = (\alpha \bar{a}) \wedge b = \alpha \bar{a} \wedge b.\]
Logo, o termo $a \wedge b$ soma $(\bar{a} \wedge b)$ no coeficiente de $\alpha^1$
no polinômio resultante. Se, além de $a$, $b$ também é um ponto no infinito, temos que
$b = \alpha \bar{b}$. Neste caso, o termo $a \wedge b$ contribui $\bar{a} \wedge \bar{b}$
para o coeficiente de $\alpha^2$.

Logo, podemos calcular este polinômio em $\alpha$. Como queremos apenas obter
o sinal da expressão para um $\alpha$ suficientemente grande, basta devolver
o sinal \emph{do coeficiente não nulo da maior potência}.

Como o teste de pertencimento do ponto ao círculo também é um determinante,
o mesmo argumento pode ser feito. De fato, no código, minha função $\texttt{e\_in\_circle6}$
chama 4 vezes a função $\texttt{e\_area2}$, para depois apenas devolver o
sinal do coeficiente não nulo da maior potência.

Como os determinantes que estou trabalhando tem ordem pequena, os polinômios
tem tamanho constante --- no máximo grau $4$. Apesar disso, creio que esta
abordagem é o grande gargalo da minha implementação. Se as coordenadas da
entrada atingem um valor $C$, as contas que faço para testar envolvem grandezas
da ordem $C^4$. Logo, se minha entrada tiver coordenadas maiores do que
$2^{16} = 65536$, meu algoritmo precisa de mais do que 64 bits para representar
as grandezas com que opera. Logo, se tiver coordenadas com valor absoluto maior do que
$7 \cdot 10^4$, meu algoritmo necessariamente precisa utilizar precisão arbitrária,
o que o torna bem menos eficiente.

Contudo, posso garantir que os cálculos estão corretos, e que não há erros
de precisão.

\section{Animação}

Animei dois aspectos do algoritmo, ambos relacionados à estrutura de dados. O primeiro,
mais direto, é de representar a solução parcial que tenho, garantindo que os triângulos
que estão na minha solução aparecem na figura. A segunda é mostrar os triângulos que
considero na parte de localização, para animar o aspecto mais ``local'' do algoritmo.

Estas duas abordagens simples ressaltam dois aspectos interessantes do algoritmo.
O primeiro é de mostrar a solução parcial e seu desenvolvimento no decorrer do
algoritmo, e a segunda é ressaltar o quanto de tempo é investido na parte de
localização do ponto na triangulação.

\section{Eficiência}

Para ter noção do funcionamento do meu algoritmo, o comparei à uma implementação
da triangulação de Delaunay disponível no pacote Scipy. Testando com potências de
dois, fui consistentemente 360 vezes mais devagar do que a implementação deles em
entradas geradas aleatoriamente. Imagino que, destes 360, 20 sejam por estar
trabalhando com Python em si, o que me faz ter listas ligadas e tabelas de hash
em momentos onde eu teria um vetor em C. Já o fator multiplicativo de 18
\end{document}
